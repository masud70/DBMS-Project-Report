\section{Software Deployment}\label{sec:sd}

To use the system, one must be familiar with the internet. The system is available through the internet from any device with a browser. In the browser, if one enters the website address that is \texttt{student.cu.ac.bd} of the system, the browser will redirect to the landing page of the website. The landing page, contains a basic form of two fields which are \textit{ID} and \textit{Password}. Every student will be provided with a distinct ID for academic identity purposes right at the time of being admitted to the university, and with that, he will be able to log in to the system. Similarly, all the teachers and admins will be provided with their own IDs when they are appointed to the role. There are no sign-up option in the website since all the information of newly admitted students and newly appointed teachers will be added to database by an admin from the authority. So only the permitted or authorized students and teachers will be able to use the system.

After completing the login form, the user will be led to a dashboard based on his role. For a student role, one will get three options which are \textit{Payments}, \textit{Attendance}, \textit{Profile Update} on the dashboard. A teacher will have \textit{Attendance}, \textit{Profile Update} option on the dashboard. An admin however, will have access to add a new Student, a new teacher, update any information of a student or teacher, add a new fee, see and analyze the statistical data of attendance or payments and so on. An admin will also be able to add or update a payment for a specific department or a group of students of that department or even for a specific student. The payment will be automatically added to the profile of those students. Students will be able to see all the payment lists that are to be paid and all the payment lists that have already been paid from their page. They will be able to download and print the receipt of any payment from their profile which will obviously contain the payment status that is \textit{paid} or \textit{unpaid}.

In order to certify attendance, teachers will be able to create a class session, and the system will produce a temporary code for students to use. As a temporary measure, the student will find a class session on their account, and if they enter the code they received in that session into that class throughout the class period, the system will automatically verify his or her attendance. If the teacher and administration want it, the system will display all of the statistical data that has been generated automatically.

The system will have two types of admin which we primarily named as \textit{Super Admin} and \textit{Sub Admin}. When someone from the department administration or a faculty member acts as sub admin, they will only be allowed to alter or update data that pertains to his or her department such as information that belongs to student or teachers, student scholarship information etc. A super admin, on the other hand, is someone who has the authority to edit, change, or update any data in the system. A super admin will have the ability to create new administrators, add officially new appointed teachers, cancel or alter any payment fees, and see all of the statistical data provided by the system and so on. In addition, it should be emphasized that all actions performed by an admin will be logged in the database with an accurate time stamp, and a super admin will be able to see the details of that activity.

The main links to get to the required pages are located in the menubar at the top of the website, which may be accessed from any page at any time. It has a number of essential links in the footer section that will guide viewers to the university's official website, the institution's Facebook page, the notice board, and other relevant resources. 

Any time a user desires, he or she may log out of the system by clicking the \textit{Log Out} button located on the sidebar of the system. Any authorized user can log in and log out of the system any time from anywhere and that is what makes the system independence in terms of location and flexible.

\clearpage