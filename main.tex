
%===============================================================================
%                                                                              =
%===================== This document is prepared by                            =
%===================== Group 5                                                 =
%===================== With a supervision of                                   =
%===================== Dr. Rudra Pratap Deb Nath                               =
%===================== Associate Professor                                     =
%===================== Department of Computer Science and Engineering          =
%===================== University of Chittagong                                =
%                                                                              =
%===============================================================================

\documentclass[12pt, a4paper]{article}
\usepackage[utf8]{inputenc} %codification of the document

\usepackage{authblk} % This one is for adding affiliation of an author \affil command

%--------------------------
%Package for comment
\usepackage{comment}
%\usepackage[none]{hyphenat}
%-------------------------------
\usepackage{tikz}
\usepackage{calc}
%\usepackage{tabu}
%Package for multicoloumn
\usepackage{multicol}
%Package for multirow
\usepackage{multirow}
\def\checkmark{\tikz\fill[scale=0.4](0,.35) -- (.25,0) -- (1,.7) -- (.25,.15) -- cycle;} 
%----------------------------
%Package for adjust width
\usepackage{changepage}
\usepackage{array}
    \newcolumntype{P}[1]{>{\centering\arraybackslash}p{#1}}
    \newcolumntype{M}[1]{>{\centering\arraybackslash}m{#1}}

%----------------------------

\usepackage{tabto}  


%Package for coloring text
\usepackage{xcolor}

%---------------------------------

% Package for math

\usepackage{amsmath}
%------------------------------------------

% Package for images
\usepackage{float}
\usepackage{graphicx}
\graphicspath{ {./images/} }
\usepackage{subfig}


%------------------------------------------

% Package for coding

\usepackage{listings}


%------------------------------------------
% Package for algorithms

\usepackage[ruled,vlined]{algorithm2e}


%------------------------------------------

%%For coloring and linking the reference and url. 
\usepackage{hyperref}
\hypersetup{
    colorlinks=true,
    linkcolor=blue,
    filecolor=magenta,      
    urlcolor=blue,
    pdftitle={Sharelatex Example},
    %bookmarks=true,
    %pdfpagemode=FullScreen,
}
%%---------------------------------
%% A macro is a shorthand for a more complicated sequence of commands. Later in the text, the macro can be used instead of those sequence of commands. 

\newcommand {\pr}{\textit{Projection,}$\pi$}
\newcommand {\se}{\textit{Selection,} $\sigma$}
\newcommand {\cp}{\textit{Cartesian product,}$\times$}
\newcommand {\un} {\textit{Union,}$\cup$}
\newcommand {\di}{\textit{Set difference,}$-$}
\newcommand {\re}{\textit{Rename,}$\rho$}



%----------------------------

%\begin{center}
%A thesis submitted to the Technical Faculty of IT and Design at Aalborg University (AAU) and the Department of Service and Information System Engineering at Universitat Politècnica de Catalunya (UPC), in partial fulfillment of the requirements within the scope of the IT4BI-DC programme for the joint Ph.D. degree in Computer Science. The thesis is not submitted to any other organization at the same time.
%\end{center}


%-------------------------


%-------------------------------
\begin{document}

\begin{titlepage}

\begin{figure}
	\centering
	\begin{minipage}[b]{0.15\textwidth}
		\includegraphics[width=1\textwidth]{images/cu}
		%\caption{Black Image}
	\end{minipage} \hfill
	\end{figure}
	
\noindent%
  \begin{tabular}{@{}p{\textwidth}@{}}
    \hline
    \hline
    \vspace{0.2cm}
    \begin{center}
    \Huge{\textbf{
      Project Title % insert your title here
    }}
    \end{center}
    \vspace{0.2cm}\\
    \hline
    \hline
  \end{tabular}
  \vspace{4 cm}

\begin{center}
    {\large
      Database Project Report %Insert document type (e.g., Project Report)
    }\\
    \vspace{0.2cm}
    {\Large
      Group-05 % Insert group number
      
    }
  \end{center}
  \vfill
  
  \begin{center}
  Report submitted December 26, 2021
  \end{center}
	\vfill
A project submitted to Dr. Rudra Pratap Deb Nath, Associate Professor, Department of Computer Science and Engineering, Chittagong University (CU) in partial fulfillment of the requirements for the Database Systems Lab course. The project is not submitted to any other organization at the same time. 

\end{titlepage}
\clearpage
%%%% Details of student
%-------------------------
%Fill up the table with your group information
%----------------------------

\begin{table}[t]
\caption{Details of Group-05}% insert your group number 
\resizebox{\textwidth}{!}{%
\begin{tabular}{|l|l|l|l|l|}
\hline
Roll / ID & Name & Sigature & Date & Supervisor Approval \\ \hline
19701070 & Md. Masud Mazumder & & &\\ \hline
19701066 & Tonmoy Chandro Das & & &\\ \hline
19701068 & Tareq Rahman Likhon Khan & & &\\ \hline
19701040 & Hamza Mohtadee Ibne Mamun & & &\\ \hline
19701019 & Polash Hossen & & &\\ \hline
187010** & Nu-Sai Mong Marma & & &\\ \hline    
\end{tabular}%
}
\end{table}
\clearpage






%-------------------------

% Showing contents as an Index
\setcounter{secnumdepth}{5}
\setcounter{tocdepth}{5}
\tableofcontents
\listoffigures

\listoftables
\lstlistoflistings
%-----------------------




\begin{abstract}
The University of Chittagong’s online payment and attendance system is the objective of this project report. A large number of students at the university pay all university fees using bank drafts to the institution’s accounts at a particular bank branch that does not allow for the use of internet payment methods. Moreover, a manual attendance system is still being practised here. Both of these analogue systems are inefficient. Particularly during examination seasons, when the majority of students are required to pay examination fees. It is marked by lengthy lines, excessive waiting on the part of students, and congestion at the banks where payments are made throughout this procedure. On the other hand, the current manual attendance system consumes a significant amount of time every day. Against this backdrop, we began work on a project to create an alternative payment and attendance system that would allow students to pay and show up for the class online. This method ensures that all students are acquainted with the online payment processes. Additionally, taking attendance online will save time, and classes will be more effective. 


To maintain the system development process, we used the Software Development Life Cycle (SDLC) and the Scrum method to help team members work together. The system was developed using a JavaScript-based framework called ”React”, which includes Cascading Style Sheets(CSS) for the front-end and ”ExpressJs” for the back-end, as well as an Apache web server and a MySQL database server. Testing and validation of the system were also carried out by enabling users to engage with it while interacting with test data. For the time being, the system is solely capable of handling the payment and attendance systems. Our system can develop this system in the future to include numerous online systems such as the No Objection Certificate (NOC), the Student Management System, the Employee Management System, and so on. The project’s outcome is an online payment and attendance system for the University of Chittagong, which alleviates the long-standing challenges associated with the university’s present ways of payment and the time-consuming manual attendance method.

%Explain the following points: Why are you doing this database project? What is the problem you choose? Why does it motivate you? What are current problems faced by the stack-holders? What solution will your system provide? What are the process you will use to develop your solution? The significance of your project, limitation and future work in short%


\end{abstract}
%\clearpage
% Maintain the consistency.
% Maintain a good writing flow. 

\section{Introduction}\label{sec:introduction}
Explain from abstract level:
Write why you are doing this project and writing this document. 

The objective of this course is to develop a database application system by applying the theories, methodologies, tools, and technologies we learnt in [write database course theory code and name].  




\subsection{Background and Motivation}\label{subsec:bm}
Write the background and motivation of project. What is the current state of the problem? What are the problems currently faced by the stack holders? What is your approach to solve/address the problems? Write the significance of your solution.   

\subsection{Problem Statement}\label{subsec:ps} Precisely state your problem statement, i.e., what is the problem and what you are going to address. Technically mention the entity types or relationships in the statement

\subsection{System Definition}\label{subsec:sd} 
Also write a system definition: A concise description of a computerized system ( that you are about to develop) expressed in natural language

A system definition example of a Conference planning system

\textit{``A computerized system used to control the ICCIT conference by registering participants and their payments to organizers using invoicing and other reporting methods. Controlling should be easy to learn, as ICCIT conferences use unpaid and untrained labor."}


\subsection{System Development Process}\label{subsec:sdp}
Write the system development process. Try to use a figure to describe the process. In brief, the  steps are 1) Requirement gathering and analysis, 2) Database modeling: conceptual modeling, logical modeling, and normalization, 3) System architecture, 4) Implementation, and 5) Validation. Briefly describe each step. Remember the output of a step is the  input of the immediate next step. Write that each step of the system development process will be a separate section of this document.
\begin{comment}


To design a database, one should follow the following steps:
\begin{enumerate}
\item Requirement analysis
	\begin{itemize}
		\item[-] interviewing, documentation, etc .
	\end{itemize}

\item Mapping onto a conceptual model (conceptual design)
     \begin{itemize}
     	\item[-] ER model
     \end{itemize}
\item Mapping onto a data model (logical design)
	\begin{itemize}
     	\item[-] Relational model, object model etc. 
     \end{itemize}
\item Normalization
\item System Architecture
\item Realization and Implementation (physical design)    
    
\end{enumerate}
\end{comment}


\subsection{Organization} Write the organization of the document here. For example: Section~\ref{sec:introduction} gives the overview of the project, Section~\ref{sec:projectmanagement} describes how the project and the resources are managed. .........Finally, the conclusion and the pointers to the future work are outlined in Section~~\ref{sec:cfw}.

\clearpage
\section{Project Management}\label{sec:projectmanagement}
To implement the system we choosed web platform since it will be easier to use with any kind of devices. It will requires only a browser and internet to use the application and one can use it from anywhere at anytime he wishes to. The application will require a frontend which is to interact with it's user through browser and backend which will perform functionalities and some will make some operations into the database based on the query that is selected by the user.\\
Developing this application would requires a huge time if we would have wished to complete it individually. So we initially planned to split the entire project into certain branches with small or medium scaled tasks and distribute them among all of our teammates. After the completion we planned to assemble all the branches from our teammates and merge them together to build the final system successfully. It is clearly visible that we were moving forward with Scrum\footnotemark method while completing the project. Following scrum model provision we made following 5 steps or phases mentioned below to execute our project. 
\begin{enumerate}
\item \textbf{Product Backlog Creation}\\
In this process, we transformed the significance and functional details of the system into short stories\\ 
If we think of a regular student being whose semester final exam is in a few days, has to pay the semester fee in a specific time period set by the administrator. This causes students suffering and directly effects on their preparation. Besides, in conventional way of transaction, university is fully dependent on one specific bank branch which causes an exhausting and tormenting situation when a huge number of students wishes to pay the fees on the same day. Many students even fail to submit the fee on time as they can't make any transaction on non-working hours. Teachers are also recorded to lose about one-third of the course period in taking attendance manually which effects on the concentration of students. On the other hand administration often find it tiresome to update data of students manually in papers. It not only consumes huge time, but also requires physical presence of a good number of person to complete. \\
These reasons influenced our team to develop a single application with the knowledge of our database course that is able to solve all of the issues mentioned above.


\item \textbf{Sprint planning and creating backlog}\\

The next stage was to do the sprint backlog creation for which our team had to select the important user stories and made them into smaller tasks. We made plans on how to get the task completed. Also, our team did prioritize the necessary tasks to accomplish the project more smoothly. The sprint duration lasted about 9 to 10 days which was long enough to allow the developer to work thoroughly. 

\item \textbf{Working on sprint}\\

This was a practical phase. The actual stories were moved as small tasks in the sprint backlog where the actual work started. To begin with, a task board also called Kanban board in Trello\footnotemark was made with a lot of cards in used. The cards used to specify the details about the tasks such as assignee, work details, due date or the time duration, etc. This cards also were used to record our meeting schedule and details as well as future updates. The task board consisted of the following columns the "To Do" lists, "Work In Progress" and then "Testing" and "Work Done" columns. A typical task board is shown in the diagram below.
\\

In scrum method, a project is usually done through merging the contributions of a whole team being led a leader among them. In our case, Md Masud Majumdar was the leader among us, who planned and led the team from beginning to end. Each team member had contributed in parts of the entire project.\\

We held regular virtual meetings where all the teammates together, came up with all the various ideas to design the database schema. The meetings we held virtually through Google Meet\footnotemark and Zoom software were led by Tareq Rahman Khan. Within a week, we decided to draw the first E-R Diagram on our ideas which was fianlly done by Palash Hossen. Later it was modified by Nu Sai Mong Marma.\\

While developing the system, we used Github\footnotemark to make the collaboration among teammates and to integrate all the assigned task and make the final system. Github not only allowed us to collaborate with our teammates but also to control the modifications of the application. Since the start of the project, our team made about 100 of commits into Github repository. All the commits are from Tonmoy Chandro Das and Md. Masud Majumdar as frontend and backend funcionalities were developed by them. To interact with github, we used git in our local machine and Visual Studio Code as code editor.\\

While developing the system, we also looked after about the documentation of this entire project. This documentation was prepared under the leadership of Hamza Mohtadee.\\

\begin{figure}[H]
    \centering
    \includegraphics[width=1\textwidth]{images/scrum_board}
    \caption{Project Management Agile Scrum Board Template}
    \label{fig:scrum_board}
\end{figure}

\item \textbf{Testing and Product Demonstration}\\

This phase is basically the modification phase based on the review of stack holders of our system. The tasks completed were to be realized as a working product with full life cycle testing. Every sprint that was completed was demonstrated to the stack holders i.e. students, teachers and concerned authority for their acceptance and their viewpoint on the complete solution. We received cordial reviews from the stack holders and in most of the cases they expressed their pleasures on the solution given by us. Being asked for suggestion, a part of them gave us some suggestion for modifications on several parts of the application. We obviously implemented many of those and integrated to the final system.

\item \textbf{Retrospective and the next sprint planning}\\

The result of this step was to discuss what had gone well and what could be improved for the next level. Also we discussed the lessons learned and the pitfalls of any particular issues or problems related to this project. We planned our next sprint that is future work. We aim to integrate more functions into the system and withing a time, make it a reliable fully functioned university management system. With the current knowledge and more study, our team is quite confident to make that possible. This will results a paperless administration. All the official tasks will require less time than ever. In the meantime, we will assuredly maintain the system to make it an uninterrupted service provider application.


\end{enumerate}

From the beginning of the planning to the final execution of the system, each and every member of our team has equally likely contributed to the project. We tried to make a better system than current one and it was finally possible as together all of us went through hundreds of ideas. We tried to make the best use of current technologies to complete this project. During the progress, we learned a lot of new things while being stuck on problems we faced never before. Without proper project management principles, projects could be handled haphazardly and are at a much higher risk of project failure, delay in the project, and being over budget. Knowing the fundamentals of project management improved our chances of completing a project successfully. 

\footnotetext[1]{Scrum is an agile development methodology used in the development of Software based on an iterative and incremental processes. Scrum is adaptable, fast, flexible and effective agile framework that is designed to deliver value to the customer throughout the development of the project.}

\footnotetext[2]{Trello is a web-based, Kanban-style, list-making application and is developed by Trello Enterprise, a subsidiary of Atlassian. It is a visual collaboration tool that enables one to organize and prioritize projects in a fun, flexible, and rewarding way. A Trello board is a series of lists, with a bunch of cards attached and packed full with powerful features and automation. [website: www.trello.com]}

\footnotetext[3]{Google Meet is a video-communication service developed by Google. It is one of two apps that constitute the replacement for Google Hangouts, the other being Google Chat. website[www.meet.google.com]}

\footnotetext[4]{GitHub is a provider of Internet hosting for software development and version control using Git. It offers the distributed version control and source code management functionality of Git, plus its own features. It lets us and others work together on projects from anywhere. [website: www.github.com]}

\clearpage
\section{Requirement Gathering and analysis}\label{sec:rga}
Explain how you gather the requirements of your problem: documentation, interviewing, survey, discussion, etc. 
Who are the stack-holders of your system?
\clearpage
\section{Conceptual Modelling}\label{sec:cm}

Data modeling is the practice of using words and symbols to represent data and how it flows in a simplified representation of a software system and the data pieces it includes. Data models serve as a roadmap for creating a new database or reengineering an existing one. A data model is a flowchart that depicts data items, their properties, and the relationships that exist between them. Before any code is developed, it allows data management and analytics teams to establish data needs for apps and uncover mistakes in development plans.\\
Data models can generally be divided into three categories, which vary according to their degree of abstraction. The process will start with a conceptual model, progress to a logical model and conclude with a physical model. We describe conceptual data modeling in this section.\\

In software development, conceptual data modeling is a concept that represents the relationships between different entities in a database. This data model is generated as a result of the requirement analysis and is the most basic type of data modeling. Therefore, this paradigm, which is known as extremely abstract, is easy to understand by any technician or non-technical person. In this data model, the entity's attributes may or may not be present. Business stakeholders and data architects are frequently involved in the creation of this data model. 
We analyzed our project requirements and developed a very basic data model (Conceptual Data Model), which is shown below.

\begin{figure}[H]
    \centering
    \includegraphics[width=1\textwidth]{images/conceptual}
    \caption{Conceptual Data Model of CU-OPAS}
    \label{fig:archi}
\end{figure}

Conceptually model your database using an E-R diagram. Use the legends in your diagram. Write how you find the entity types, relationships, and attributes from Section~\ref{sec:rga}
\section{Logical Modelling}\label{sec:lm}
The logical data model (LDM) is the conceptual data model's enlarged format. It describes how to set up a system that is not particular to any database. It primarily establishes the data elements and the relationships between them. A logical data model is the foundation of the physical data model (FDM). Attributes, primary keys, foreign keys, relationship cardinality, and descriptive entities and classes are all described in an LDM. This data model clearly defines all of the relationships between the entities. As a result, anyone may convert an LDM to an FDM in any database management system. This data model is often created by data architects and business analysts.\\

\begin{figure}[H]
	\captionsetup[subfigure]{labelformat=empty}
	\hfill
		\subfloat{\includegraphics[width=0.7\textwidth]{images/logical}}
	\hfill
		\subfloat{\includegraphics[width=0.3\textwidth]{images/legend}}
	\hfill
	\caption{Logical Data Model of CU-OPAS}
\end{figure}
Write a short description of Relation model. 
Write a how you convert your E-R model in Relational model
\clearpage
\section{Normalization}\label{sec:norm}
For less complexity we are going to rename the attributes with uppercase letters here. Following the statement,\\
let,
\begin{multicols}{2}
\begin{adjustwidth}{2cm}{}

\noindent 
\textbf{A} = \textit{user\_id}\\
\textbf{B} = \textit{name}\\
\textbf{C} = \textit{email}\\
\textbf{D} = \textit{role}\\
\textbf{E} = \textit{student\_id}\\
\textbf{F} = \textit{session}\\
\textbf{G} = \textit{hall}\\
\textbf{H} = \textit{teacher\_id}\\
\textbf{I} = \textit{rank}\\
\textbf{J} = \textit{faculty\_name}\\
\textbf{K} = \textit{dean}\\
\textbf{L} = \textit{department\_name}\\
\textbf{M} = \textit{semester\_name}\\
\vfill\null
\columnbreak

\noindent
\textbf{N} = \textit{course\_name}\\
\textbf{O} = \textit{course\_code}\\
\textbf{P} = \textit{purpose\_title}\\
\textbf{Q} = \textit{created\_by}\\
\textbf{R} = \textit{payment\_details}\\
\textbf{S} = \textit{payment\_amount}\\
\textbf{T} = \textit{due\_date}\\
\textbf{U} = \textit{payment\_date}\\
\textbf{V} = \textit{payment\_method}\\
\textbf{W} = \textit{class\_date\_time}\\
\textbf{X} = \textit{class\_is\_active}\\
\textbf{Y} = \textit{class\_teacher}\\
\textbf{Z} = \textit{attendance\_status}\\
\end{adjustwidth}

\end{multicols}

The Boyce-Codd Normal form of\\

\begin{adjustwidth}{2cm}{}
\textbf{R(A, B, C, D, E, F, G, H, I, J, K, L, M, N, O, P, Q, R, S, T, U, V, W, X, Y, Z)}\\
Functional Dependency:\\
\{\\
\textit{
A → B C D\\
E → F G\\
H → I\\
J → K\\
N → O\\
P → Q\\
R → S\\
W → X Y\\
} 
\}\\
\end{adjustwidth}
is:

\begin{multicols}{2}
\begin{adjustwidth}{2cm}{}

\textbf{R\textsubscript{11} (J, K)}\\
FD:\\
\{\textit{ J → K }\} \\ \\

\noindent
\textbf{R\textsubscript{13}(H, I)}\\
FD:\\
\{ \textit{H → I} \} \\ \\

\noindent
\textbf{R\textsubscript{15} (E, F, G) }\\
FD:\\
\{\\
\textit{ 
E → F\\
E → G\\
}
\} \\ \\
	   
\noindent	   
\textbf{R1\textsubscript{16} (A, E, H, J, L, M, N, P, R, T, U, V, W, Z)}\\
FD:\\ \{ \}\\ \\

\noindent
\textbf{R\textsubscript{1} (A, B, C, D)}\\
FD:\\
\{\\
\textit{ 
A → B\\
A → C\\
A → D\\
}
\} \\ \\

\noindent
\textbf{R\textsubscript{3} (W, X, Y)}\\
FD:\\
\{\\
\textit{ 
W → X\\
W → Y\\ 
}
\} \\ \\

\noindent
\textbf{R\textsubscript{5} (R, S)}\\
FD:\{
\textit{ 
R → S 
}
\} \\ \\

\noindent
\textbf{R\textsubscript{7} (P, Q)}\\
FD:\{
\textit{ 
P → Q 
}
\} \\ \\

\noindent
\textbf{R\textsubscript{9} (N, O)}\\
FD:\{
\textit{ 
N → O 
}
\} \\ \\

\end{adjustwidth}
\end{multicols}

The dependencies are preserved.\\

Here is the transcript of the algorithm:\\

In:\\

\begin{adjustwidth}{2cm}{}
\textbf{R (A, B, C, D, E, F, G, H, I, J, K, L, M, N, O, P, Q, R, S, T, U, V, W, X, Y, Z)}\\
FD: \\
\{ \\
\textit{
A → B\\
A → C\\
A → D\\
E → F\\
E → G\\
H → I\\
J → K\\
N → O\\
P → Q\\
R → S\\
W → X\\
W → Y\\
}
\}\\
\end{adjustwidth}
 
The determinant of A → B is not a superkey and so R is replaced by:\\

\begin{adjustwidth}{2cm}{}
\textbf{R\textsubscript{1} (A, B, C, D)}\\
\{\\ 
\textit{
A → B\\
A → C\\
A → D\\
} 
\}\\
\end{adjustwidth} 
and:\\ 

\begin{adjustwidth}{2cm}{}
\textbf{R\textsubscript{2} (A, E, F, G, H, I, J, K, L, M, N, O, P, Q, R, S, T, U, V, W, X, Y, Z)}\\
FD:\\ 
\{ \\ 
\textit{
W → X\\
W → Y\\
R → S\\
P → Q\\
N → O\\
J → K\\
H → I\\
E → F\\
E → G\\
}
\}\\ \\
\end{adjustwidth} 

In:\\

\begin{adjustwidth}{2cm}{}
\textbf{R\textsubscript{2} (A, E, F, G, H, I, J, K, L, M, N, O, P, Q, R, S, T, U, V, W, X, Y, Z)}\\
FD:\\
\{ \\
\textit{  
W → X\\
W → Y\\
R → S\\
P → Q\\
N → O\\
J → K\\
H → I\\
E → F\\
E → G\\
}
\}\\
\end{adjustwidth} 

the determinant of W → X is not a superkey and so R\textsubscript{2} is replaced by:

\begin{adjustwidth}{2cm}{}
\textbf{R\textsubscript{3} (W, X, Y)}\\
FD:\\
\{\\
\textit{ 
W → X\\
W → Y\\ 
}
\} \\
\end{adjustwidth} 

and:\\

\begin{adjustwidth}{2cm}{}
\textbf{R\textsubscript{4} (A, E, F, G, H, I, J, K, L, M, N, O, P, Q, R, S, T, U, V, W, Z)}\\
FD:\\ 
\{\\ 
\textit{ 
R → S\\
P → Q\\
N → O\\
J → K\\
H → I\\
E → F\\
E → G\\
} 
\}\\
\end{adjustwidth} 

in:\\

\begin{adjustwidth}{2cm}{}
\textbf{R\textsubscript{4} (A, E, F, G, H, I, J, K, L, M, N, O, P, Q, R, S, T, U, V, W, Z)}\\
FD:\\ \{\\ 
\textit{ 
R → S\\
P → Q\\
N → O\\
J → K\\
H → I\\
E → F\\
E → G\\
} 
\}\\
\end{adjustwidth} 

the determinant of R → S is not a superkey and so R\textsubscript{4} is replaced by:\\

\begin{adjustwidth}{2cm}{}
\textbf{R\textsubscript{5} (R, S)}\\
FD:\{
\textit{ 
R → S 
}
\} \\
\end{adjustwidth} 

and:\\

\begin{adjustwidth}{2cm}{}
\textbf{R\textsubscript{6} (A, E, F, G, H, I, J, K, L, M, N, O, P, Q, R, T, U, V, W, Z)}\\
FD:\\
\{\\
\textit{ 
P → Q\\
N → O\\
J → K\\
H → I\\
E → F\\
E → G \\
}
\} \\
\end{adjustwidth}

the determinant of P → Q is not a superkey and so R\textsubscript{6} is replaced by:\\

\begin{adjustwidth}{2cm}{}
\textbf{R\textsubscript{7} (P, Q)}\\
FD:\{
\textit{ 
P → Q 
}
\} \\
\end{adjustwidth} 

and:\\

\begin{adjustwidth}{2cm}{}
\textbf{R\textsubscript{8} (A, E, F, G, H, I, J, K, L, M, N, O, P, R, T, U, V, W, Z)}\\
FD:\\
\{\\
\textit{ 
N → O\\
J → K\\
H → I\\
E → F\\
E → G\\
}
\} \\
\end{adjustwidth}

In:\\

\begin{adjustwidth}{2cm}{}
\textbf{R\textsubscript{8} (A, E, F, G, H, I, J, K, L, M, N, O, P, R, T, U, V, W, Z)}\\
FD:\\
\{\\
\textit{ 
N → O\\
J → K\\
H → I\\
E → F\\
E → G\\
}
\} \\
\end{adjustwidth}

the determinant of N → O is not a superkey and so R\textsubscript{8} is replaced by:\\

\begin{adjustwidth}{2cm}{}
\textbf{R\textsubscript{9} (N, O)}\\
FD:\{
\textit{ 
N → O 
}
\} \\
\end{adjustwidth} 

and:\\

\begin{adjustwidth}{2cm}{}
\textbf{R\textsubscript{10} (A, E, F, G, H, I, J, K, L, M, N, P, R, T, U, V, W, Z)}\\
FD:\\
\{\\
\textit{ 
J → K\\
H → I\\
E → F\\
E → G\\
}
\} \\
\end{adjustwidth} 

the determinant of J → K is not a superkey and so R\textsubscript{10} is replaced by:\\

\begin{adjustwidth}{2cm}{}
\textbf{R\textsubscript{11} (J, K)}\\
FD:\{
\textit{ 
J → K 
}
\} \\
\end{adjustwidth} 

and:\\

\begin{adjustwidth}{2cm}{}
\textbf{R\textsubscript{12} (A, E, F, G, H, I, J, L, M, N, P, R, T, U, V, W, Z)}\\
FD:\\
\{\\
\textit{ 
H → I\\
E → F\\
E → G\\
}
\} \\
\end{adjustwidth} 


In:\\

\begin{adjustwidth}{2cm}{}
\textbf{R\textsubscript{12} (A, E, F, G, H, I, J, L, M, N, P, R, T, U, V, W, Z)}\\
FD:\\
\{\\
\textit{ 
H → I\\
E → F\\
E → G\\
}
\} \\
\end{adjustwidth} 

the determinant of H → I is not a superkey and so R\textsubscript{12} is replaced by:

\begin{adjustwidth}{2cm}{}
\textbf{R\textsubscript{13} (H, I)}\\
FD:\{
\textit{ 
H → I 
}
\} \\
\end{adjustwidth} 

and:\\

\begin{adjustwidth}{2cm}{}
\textbf{R\textsubscript{14} (A, E, F, G, H, J, L, M, N, P, R, T, U, V, W, Z)}\\
FD:\\
\{\\
\textit{ 
E → F\\
E → G\\
}
\} \\
\end{adjustwidth} 

In:\\

\begin{adjustwidth}{2cm}{}
\textbf{R\textsubscript{14} (A, E, F, G, H, J, L, M, N, P, R, T, U, V, W, Z)}\\
FD:\\
\{\\
\textit{ 
E → F\\
E → G\\
}
\} \\
\end{adjustwidth} 

the determinant of E → F is not a superkey and so R\textsubscript{14} is replaced by:

\begin{adjustwidth}{2cm}{}
\textbf{R\textsubscript{15} (E, F, G)}\\
FD:\\
\{\\
\textit{ 
E → F\\
E → G\\
}
\} \\
\end{adjustwidth}

and:\\

\begin{adjustwidth}{2cm}{}
\textbf{R\textsubscript{16} (A, E, H, J, L, M, N, P, R, T, U, V, W, Z)}\\
FD:\{\} \\ \\
\end{adjustwidth}

The final results are:\\

\begin{adjustwidth}{2cm}{}
\textbf{R\textsubscript{11} (J, K)}\\
FD:\{
\textit{ 
J → K 
}
\} \\ \\

\noindent
\textbf{R\textsubscript{13} (H, I)}\\
FD:\{
\textit{ 
H → I 
}
\} \\ \\

\noindent
\textbf{R\textsubscript{15} (E, F, G)}\\
FD:\\
\{\\
\textit{ 
E → F\\
E → G\\
}
\} \\ \\

\noindent
\textbf{R\textsubscript{16} (A, E, H, J, L, M, N, P, R, T, U, V, W, Z)}\\
FD:\{\} \\ \\

\noindent
\textbf{R\textsubscript{1} (A, B, C, D)}\\
\{\\ 
\textit{
A → B\\
A → C\\
A → D\\
} 
\}\\ \\

\noindent
\textbf{R\textsubscript{3} (W, X, Y)}\\
FD:\\
\{\\
\textit{ 
W → X\\
W → Y\\ 
}
\} \\ \\

\noindent
\textbf{R\textsubscript{5} (R, S)}\\
FD:\{
\textit{ 
R → S 
}
\} \\

\noindent
\textbf{R\textsubscript{9} (N, O)}\\
FD:\{
\textit{ 
N → O 
}
\} \\


\end{adjustwidth} 




\clearpage
\section{Physical Modelling}\label{sec:phy}
Physical Data Modeling (PDM) is the final step in data modeling. A PDM is primarily concerned with the implementation of a database-specific data model. As a result, it depicts how the model will be created in the database. For a non-technical individual, this data model is difficult to comprehend. This data model is required to create a query for CRUD activities.
We created a Physical Data Model based on Logical Data Modeling (LDM) and Normalization. We utilized the MYSQL database server and applied the FDM reconsideration to it in this case. The name convention for entities and attributes was "Snake Case." For example, we used "student id" for the attribute name "Student ID." The Physical Data Model we used in our MYSQL database server is shown below.\\

\begin{figure}[H]
    \centering
    \includegraphics[width=1\textwidth]{images/physical}
    \caption{Physical Data Model of CU-OPAS}
    \label{fig:physical}
\end{figure}

\clearpage
\section{System Architecture}\label{sec:sa}
Due to the fact that it is a web platform, the system is divided into two parts: the front-end and the back-end. Front-end refers to the component of the system that will be visible to the user on the screen of a mobile device or computer. Whereas the back end refers to parts of the system or a program's code that allow it to operate and that cannot be accessed by a user. The back end is also called the data access layer of a system and includes any functionality that needs to be accessed and navigated to by digital means. The front-end is designed with more than 25 libraries including leading front-end library \textit{React.js}, \textit{MaterialUI}, \textit{Ant-Design}, \textbf{Bootstrap} and so on. These libraries are written on \textit{Javascript} programming language and is able to be integrated with any other programming language based back-end libraries. We used more than 15 Javascript based frameworks such as \textit{Express Js}. For database management, we used \textit{MySQL}\cite{mysql} which is an open-source relational database management system. When a user tries to log in to the system with credentials, the the system catches the input data and send those to the back-end first for certifying the authorized user. Back-end functions then connect with the database server, and certifies the user.

All of the front-end information that a user sees is retrieved from the database and sent to the browser using back-end functionalities. On the dashboard, different options are shown based on the roles of the users. CRUD (Create, Read, Update, Delete) operations, that may be performed on the selections of the user from the dashboard. API\cite{api} (\textit{Application Programming Interface}) allows the back-end to retrieve and alter data into the database through SQL queries. The following figure is a diagrammatic representation of the work-flow of the system.

Figure 13 illustrates the work flow of CU-OPAS(Chittagong University Online Payment and Attendance System). From the figure we can see that he system validates an user with his login credentials and shows different dashboard for students and teachers.

\begin{figure}[H]
    \centering
    \label{fig:flowchart}
    \includegraphics[height=15cm, width=1\textwidth]{images/flowchart}
    \caption{Work flow of the system}
\end{figure}

\clearpage
\section{Implementation}\label{sec:imp}
Give some code snippet of each component you outlined in your System architecture. Some DDL query example. Use the listing environment for writing code. 
Listing~\ref{list:sql} shows an SQL query. 

\begin{lstlisting}[caption={A SQL query example}, label=list:sql, captionpos=b,
           backgroundcolor=\color{white},
           language=SQL,
           breaklines=true,
           frame=single,
           showspaces=false,
           basicstyle=\ttfamily,
           numbers=left,
           numberstyle=\tiny,
           rulecolor=\color{red},
           keywordstyle=\color{blue},
           commentstyle=\color{gray}
        ]
select distinct name
from instructor
where salary > some( select salary
			from instructor
			where dept_name='CSE');
\end{lstlisting}
\clearpage

\section{Validation} \label{sec:val}
Show that users are satisfied with your product. 
You can also give a user manual here describing how to use your system (process of completion of different tasks using your system )
You can use some matrices (time, cost, resource etc.) to compare your system with the previous system. 
\clearpage
\section{Software Deployment}\label{sec:sd}
Describe how to install and configure your system so that a non-technical user can use your system. 

\clearpage
\section{Conclusion and Future Work}\label{sec:cfw}

\clearpage

\section{Bibliography} 
\label{sec:bibliography}
To add bibliography in your document, use the following steps:
\begin{enumerate}
\item First create a .bib file in the same directory where your .tex file is (in our case, the file name is references.bib). Also place the bibliography style file in the same directory. In our case, we are using the ios1.bst style file. We include the following commands in the .tex file for the style file and bib file: \\
 \texttt{\textbackslash bibliographystyle\{ios1\}} \\
\texttt{\textbackslash bibliography\{references\}} 
\item Import the BibTeX of your book or paper from Google Scholar or other sources into your .bib file. An example of BibTex is shown in Listings~\ref{list:bibtex}.  

\begin{lstlisting}[caption={A BibTeX example}, label=list:bibtex, captionpos=b,
           backgroundcolor=\color{white},
           language=SQL,
           breaklines=true,
           frame=single,
           showspaces=false,
           basicstyle=\ttfamily,
           numbers=left,
           numberstyle=\tiny,
           rulecolor=\color{red},
           commentstyle=\color{gray}
        ]
@article{kopka1995guide,
  title={A Guide to $\{$$\backslash$LaTeX$\}$--Document},
  author={Kopka, H and Daly, PW},
  year={1995},
  publisher={Citeseer}
}
\end{lstlisting}

\item Then, use the name of the BibTex (in Listing~\ref{list:bibtex}, the name is kopka1995guide) in the text of your .tex document where you want to refer it.

\item After saving your .tex document, execute the PDFLaTeX option one time; then execute the BibTeX option; then again execute the PDFLaTeX option for twice; finally, execute the QuickBuild option. Now your document refer the corresponding book or paper. 
\end{enumerate}

%\bibliographystyle{ios1}

\bibliographystyle{plain}
\bibliography{references.bib}


\end{document}