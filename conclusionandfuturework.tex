\section{Conclusion and Future Work}\label{sec:cfw}
The risks avoidance and reduction strategies that could be implemented to ensure project success is: Defining project key success factors at the beginning, then throughout the development stages, each is resolved by designing the appropriate policies, regulations, system model, and database tables design. For the development of an online payment system to handle tuition fees as well as other payments and attendance systems to reduce the time consumption as well as to make attendance taking procedure more flexible and more transparent which integrates the current university system, we started with the enlisting the issues and difficulties faced by both students and administration while using the current analog system and then came up with the various ideas to solve these problems. While choosing the best idea among those, we took into consideration the system being uncomplicated to use, transparent, flexible, being unbounded from time, physical presence, being usable with multiple possible ways of transactions, region independent. The system not only offers these mentioned properties but also offers an integrated attendance system that can replace the current analog time-consuming system. Our developed system proposes a paperless attendance attesting system that is easy to use and offers more features and automatically generated statistics, usually being done manually.

In an educational institute like the University of Chittagong, where the administration has to deal with a massive number of students, it has been an exigency situation to replace the analog system with a digital one. Our system ensures the students' gratification and ensures the liberty of administration from being dependent on a specific bank as an agent to make the transaction with students. Once teachers start using the newly built attendance system, they can save even more time for teaching, as seen in the statistics. Administrations will be able to see more statistical data on both payments and attendance, which could be helpful to make important and critical decisions. The purpose of this project is to resolve the issues with the current system and replace the analog procedure of both payment and attendance with an efficient, flexible, convenient, and secured system and our developed system promises to offer these core features as we have used the latest technologies to build it.

Despite being efficient, reliable, flexible, and secured, we found two limitations in this newly developed system. Stack-holders, i.e., students, teachers, and administrations, must be experienced in using the internet as a prerequisite to use this system. Otherwise, the system may appear difficult to use sometimes. Following the prerequisite, the major limitation of this system is that it requires the internet to use it. As the system is developed as a web application, it can be accessed only via browsers. Although any devices such as mobile, tablets, and desktops with any operating system can access this system via browser, without the internet, the browser will not be able to access the system. Whereas the internet makes this system convenient and flexible, it makes the system depends on it. 

When the system was given to use primarily to experience by the students and teachers on testing purposes, they expressed their satisfaction and suggested integrating more features into it. We aim to integrate the whole student management system into it to make all the administrative operations paperless and digital. While moving forward to the coming days, the system will require more modifications and updates as the number of students as well as teachers will always increase. 

\clearpage