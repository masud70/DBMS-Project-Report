\section{Normalization}\label{sec:norm}
For less complexity we are going to rename the attributes with uppercase letters here. Following the statement,\\
let,
\begin{multicols}{2}
\begin{adjustwidth}{2cm}{}

\noindent 
\textbf{A} = \textit{user\_id}\\
\textbf{B} = \textit{name}\\
\textbf{C} = \textit{email}\\
\textbf{D} = \textit{role}\\
\textbf{E} = \textit{student\_id}\\
\textbf{F} = \textit{session}\\
\textbf{G} = \textit{hall}\\
\textbf{H} = \textit{teacher\_id}\\
\textbf{I} = \textit{rank}\\
\textbf{J} = \textit{faculty\_name}\\
\textbf{K} = \textit{dean}\\
\textbf{L} = \textit{department\_name}\\
\textbf{M} = \textit{semester\_name}\\
\vfill\null
\columnbreak

\noindent
\textbf{N} = \textit{course\_name}\\
\textbf{O} = \textit{course\_code}\\
\textbf{P} = \textit{purpose\_title}\\
\textbf{Q} = \textit{created\_by}\\
\textbf{R} = \textit{payment\_details}\\
\textbf{S} = \textit{payment\_amount}\\
\textbf{T} = \textit{due\_date}\\
\textbf{U} = \textit{payment\_date}\\
\textbf{V} = \textit{payment\_method}\\
\textbf{W} = \textit{class\_date\_time}\\
\textbf{X} = \textit{class\_is\_active}\\
\textbf{Y} = \textit{class\_teacher}\\
\textbf{Z} = \textit{attendance\_status}\\
\end{adjustwidth}

\end{multicols}

The Boyce-Codd Normal form of\\

\begin{adjustwidth}{2cm}{}
\textbf{R(A, B, C, D, E, F, G, H, I, J, K, L, M, N, O, P, Q, R, S, T, U, V, W, X, Y, Z)}\\
Functional Dependency:\\
\{\\
\textit{
A → B C D\\
E → F G\\
H → I\\
J → K\\
N → O\\
P → Q\\
R → S\\
W → X Y\\
} 
\}\\
\end{adjustwidth}
is:

\begin{multicols}{2}
\begin{adjustwidth}{2cm}{}

\textbf{R\textsubscript{11} (J, K)}\\
FD:\\
\{\textit{ J → K }\} \\ \\

\noindent
\textbf{R\textsubscript{13}(H, I)}\\
FD:\\
\{ \textit{H → I} \} \\ \\

\noindent
\textbf{R\textsubscript{15} (E, F, G) }\\
FD:\\
\{\\
\textit{ 
E → F\\
E → G\\
}
\} \\ \\
	   
\noindent	   
\textbf{R1\textsubscript{16} (A, E, H, J, L, M, N, P, R, T, U, V, W, Z)}\\
FD:\\ \{ \}\\ \\

\noindent
\textbf{R\textsubscript{1} (A, B, C, D)}\\
FD:\\
\{\\
\textit{ 
A → B\\
A → C\\
A → D\\
}
\} \\ \\

\noindent
\textbf{R\textsubscript{3} (W, X, Y)}\\
FD:\\
\{\\
\textit{ 
W → X\\
W → Y\\ 
}
\} \\ \\

\noindent
\textbf{R\textsubscript{5} (R, S)}\\
FD:\{
\textit{ 
R → S 
}
\} \\ \\

\noindent
\textbf{R\textsubscript{7} (P, Q)}\\
FD:\{
\textit{ 
P → Q 
}
\} \\ \\

\noindent
\textbf{R\textsubscript{9} (N, O)}\\
FD:\{
\textit{ 
N → O 
}
\} \\ \\

\end{adjustwidth}
\end{multicols}

The dependencies are preserved.\\

Here is the transcript of the algorithm:\\

In:\\

\begin{adjustwidth}{2cm}{}
\textbf{R (A, B, C, D, E, F, G, H, I, J, K, L, M, N, O, P, Q, R, S, T, U, V, W, X, Y, Z)}\\
FD: \\
\{ \\
\textit{
A → B\\
A → C\\
A → D\\
E → F\\
E → G\\
H → I\\
J → K\\
N → O\\
P → Q\\
R → S\\
W → X\\
W → Y\\
}
\}\\
\end{adjustwidth}
 
The determinant of A → B is not a superkey and so R is replaced by:\\

\begin{adjustwidth}{2cm}{}
\textbf{R\textsubscript{1} (A, B, C, D)}\\
\{\\ 
\textit{
A → B\\
A → C\\
A → D\\
} 
\}\\
\end{adjustwidth} 
and:\\ 

\begin{adjustwidth}{2cm}{}
\textbf{R\textsubscript{2} (A, E, F, G, H, I, J, K, L, M, N, O, P, Q, R, S, T, U, V, W, X, Y, Z)}\\
FD:\\ 
\{ \\ 
\textit{
W → X\\
W → Y\\
R → S\\
P → Q\\
N → O\\
J → K\\
H → I\\
E → F\\
E → G\\
}
\}\\ \\
\end{adjustwidth} 

In:\\

\begin{adjustwidth}{2cm}{}
\textbf{R\textsubscript{2} (A, E, F, G, H, I, J, K, L, M, N, O, P, Q, R, S, T, U, V, W, X, Y, Z)}\\
FD:\\
\{ \\
\textit{  
W → X\\
W → Y\\
R → S\\
P → Q\\
N → O\\
J → K\\
H → I\\
E → F\\
E → G\\
}
\}\\
\end{adjustwidth} 

the determinant of W → X is not a superkey and so R\textsubscript{2} is replaced by:

\begin{adjustwidth}{2cm}{}
\textbf{R\textsubscript{3} (W, X, Y)}\\
FD:\\
\{\\
\textit{ 
W → X\\
W → Y\\ 
}
\} \\
\end{adjustwidth} 

and:\\

\begin{adjustwidth}{2cm}{}
\textbf{R\textsubscript{4} (A, E, F, G, H, I, J, K, L, M, N, O, P, Q, R, S, T, U, V, W, Z)}\\
FD:\\ 
\{\\ 
\textit{ 
R → S\\
P → Q\\
N → O\\
J → K\\
H → I\\
E → F\\
E → G\\
} 
\}\\
\end{adjustwidth} 

in:\\

\begin{adjustwidth}{2cm}{}
\textbf{R\textsubscript{4} (A, E, F, G, H, I, J, K, L, M, N, O, P, Q, R, S, T, U, V, W, Z)}\\
FD:\\ \{\\ 
\textit{ 
R → S\\
P → Q\\
N → O\\
J → K\\
H → I\\
E → F\\
E → G\\
} 
\}\\
\end{adjustwidth} 

the determinant of R → S is not a superkey and so R\textsubscript{4} is replaced by:\\

\begin{adjustwidth}{2cm}{}
\textbf{R\textsubscript{5} (R, S)}\\
FD:\{
\textit{ 
R → S 
}
\} \\
\end{adjustwidth} 

and:\\

\begin{adjustwidth}{2cm}{}
\textbf{R\textsubscript{6} (A, E, F, G, H, I, J, K, L, M, N, O, P, Q, R, T, U, V, W, Z)}\\
FD:\\
\{\\
\textit{ 
P → Q\\
N → O\\
J → K\\
H → I\\
E → F\\
E → G \\
}
\} \\
\end{adjustwidth}

the determinant of P → Q is not a superkey and so R\textsubscript{6} is replaced by:\\

\begin{adjustwidth}{2cm}{}
\textbf{R\textsubscript{7} (P, Q)}\\
FD:\{
\textit{ 
P → Q 
}
\} \\
\end{adjustwidth} 

and:\\

\begin{adjustwidth}{2cm}{}
\textbf{R\textsubscript{8} (A, E, F, G, H, I, J, K, L, M, N, O, P, R, T, U, V, W, Z)}\\
FD:\\
\{\\
\textit{ 
N → O\\
J → K\\
H → I\\
E → F\\
E → G\\
}
\} \\
\end{adjustwidth}

In:\\

\begin{adjustwidth}{2cm}{}
\textbf{R\textsubscript{8} (A, E, F, G, H, I, J, K, L, M, N, O, P, R, T, U, V, W, Z)}\\
FD:\\
\{\\
\textit{ 
N → O\\
J → K\\
H → I\\
E → F\\
E → G\\
}
\} \\
\end{adjustwidth}

the determinant of N → O is not a superkey and so R\textsubscript{8} is replaced by:\\

\begin{adjustwidth}{2cm}{}
\textbf{R\textsubscript{9} (N, O)}\\
FD:\{
\textit{ 
N → O 
}
\} \\
\end{adjustwidth} 

and:\\

\begin{adjustwidth}{2cm}{}
\textbf{R\textsubscript{10} (A, E, F, G, H, I, J, K, L, M, N, P, R, T, U, V, W, Z)}\\
FD:\\
\{\\
\textit{ 
J → K\\
H → I\\
E → F\\
E → G\\
}
\} \\
\end{adjustwidth} 

the determinant of J → K is not a superkey and so R\textsubscript{10} is replaced by:\\

\begin{adjustwidth}{2cm}{}
\textbf{R\textsubscript{11} (J, K)}\\
FD:\{
\textit{ 
J → K 
}
\} \\
\end{adjustwidth} 

and:\\

\begin{adjustwidth}{2cm}{}
\textbf{R\textsubscript{12} (A, E, F, G, H, I, J, L, M, N, P, R, T, U, V, W, Z)}\\
FD:\\
\{\\
\textit{ 
H → I\\
E → F\\
E → G\\
}
\} \\
\end{adjustwidth} 


In:\\

\begin{adjustwidth}{2cm}{}
\textbf{R\textsubscript{12} (A, E, F, G, H, I, J, L, M, N, P, R, T, U, V, W, Z)}\\
FD:\\
\{\\
\textit{ 
H → I\\
E → F\\
E → G\\
}
\} \\
\end{adjustwidth} 

the determinant of H → I is not a superkey and so R\textsubscript{12} is replaced by:

\begin{adjustwidth}{2cm}{}
\textbf{R\textsubscript{13} (H, I)}\\
FD:\{
\textit{ 
H → I 
}
\} \\
\end{adjustwidth} 

and:\\

\begin{adjustwidth}{2cm}{}
\textbf{R\textsubscript{14} (A, E, F, G, H, J, L, M, N, P, R, T, U, V, W, Z)}\\
FD:\\
\{\\
\textit{ 
E → F\\
E → G\\
}
\} \\
\end{adjustwidth} 

In:\\

\begin{adjustwidth}{2cm}{}
\textbf{R\textsubscript{14} (A, E, F, G, H, J, L, M, N, P, R, T, U, V, W, Z)}\\
FD:\\
\{\\
\textit{ 
E → F\\
E → G\\
}
\} \\
\end{adjustwidth} 

the determinant of E → F is not a superkey and so R\textsubscript{14} is replaced by:

\begin{adjustwidth}{2cm}{}
\textbf{R\textsubscript{15} (E, F, G)}\\
FD:\\
\{\\
\textit{ 
E → F\\
E → G\\
}
\} \\
\end{adjustwidth}

and:\\

\begin{adjustwidth}{2cm}{}
\textbf{R\textsubscript{16} (A, E, H, J, L, M, N, P, R, T, U, V, W, Z)}\\
FD:\{\} \\ \\
\end{adjustwidth}

The final results are:\\

\begin{adjustwidth}{2cm}{}
\textbf{R\textsubscript{11} (J, K)}\\
FD:\{
\textit{ 
J → K 
}
\} \\ \\

\noindent
\textbf{R\textsubscript{13} (H, I)}\\
FD:\{
\textit{ 
H → I 
}
\} \\ \\

\noindent
\textbf{R\textsubscript{15} (E, F, G)}\\
FD:\\
\{\\
\textit{ 
E → F\\
E → G\\
}
\} \\ \\

\noindent
\textbf{R\textsubscript{16} (A, E, H, J, L, M, N, P, R, T, U, V, W, Z)}\\
FD:\{\} \\ \\

\noindent
\textbf{R\textsubscript{1} (A, B, C, D)}\\
\{\\ 
\textit{
A → B\\
A → C\\
A → D\\
} 
\}\\ \\

\noindent
\textbf{R\textsubscript{3} (W, X, Y)}\\
FD:\\
\{\\
\textit{ 
W → X\\
W → Y\\ 
}
\} \\ \\

\noindent
\textbf{R\textsubscript{5} (R, S)}\\
FD:\{
\textit{ 
R → S 
}
\} \\

\noindent
\textbf{R\textsubscript{9} (N, O)}\\
FD:\{
\textit{ 
N → O 
}
\} \\


\end{adjustwidth} 




\clearpage