% Maintain the consistency.
% Maintain a good writing flow. 

\section{Introduction}\label{sec:introduction}
One of the major benefits of using technologies is that it costs less time and instance in most of the cases to perform the same operation than with analog process. Conventionally we still perform all the official tasks through paper work. It not only consumes more time, but also makes students dependent on authority and on specific time period set by the authority. Also, there are always chances of being unsuccessful to complete the process with or without wrong information. In addition, taking attendance manually is also a time-consuming method. Our goal is to develop an application that aims to shift these two process into a digital platform which will obviously be able to perform the given tasks in the shortest time possible with no paper work. The system should be able to let the students get the liberty from the tormenting limitations of the administration and also to let the authority to make their responsibility easier. It goes without saying that the application would be a user-friendly application to both students and authority.

This document objectifies  a recording of a strategic and creative process focused on clearly outlining issues, goals as well as overview of the application representing the narrative from the beginning to the end. Any person willing to use as well as develop the system would be able to do so by the help of this documentation.

The objective of this course is to develop a database application system by applying the theories, methodologies, tools, and technologies we learnt in CSE 413 that is Database System course.  




\subsection{Background and Motivation}\label{subsec:bm}
Write the background and motivation of project. What is the current state of the problem? What are the problems currently faced by the stack holders? What is your approach to solve/address the problems? Write the significance of your solution.   

\subsection{Problem Statement}\label{subsec:ps} Precisely state your problem statement, i.e., what is the problem and what you are going to address. Technically mention the entity types or relationships in the statement

\subsection{System Definition}\label{subsec:sd} 
Also write a system definition: A concise description of a computerized system ( that you are about to develop) expressed in natural language

A system definition example of a Conference planning system

\textit{``A computerized system used to control the ICCIT conference by registering participants and their payments to organizers using invoicing and other reporting methods. Controlling should be easy to learn, as ICCIT conferences use unpaid and untrained labor."}


\subsection{System Development Process}\label{subsec:sdp}
Write the system development process. Try to use a figure to describe the process. In brief, the  steps are 1) Requirement gathering and analysis, 2) Database modeling: conceptual modeling, logical modeling, and normalization, 3) System architecture, 4) Implementation, and 5) Validation. Briefly describe each step. Remember the output of a step is the  input of the immediate next step. Write that each step of the system development process will be a separate section of this document.
\begin{comment}


To design a database, one should follow the following steps:
\begin{enumerate}
\item Requirement analysis
	\begin{itemize}
		\item[-] interviewing, documentation, etc .
	\end{itemize}

\item Mapping onto a conceptual model (conceptual design)
     \begin{itemize}
     	\item[-] ER model
     \end{itemize}
\item Mapping onto a data model (logical design)
	\begin{itemize}
     	\item[-] Relational model, object model etc. 
     \end{itemize}
\item Normalization
\item System Architecture
\item Realization and Implementation (physical design)    
    
\end{enumerate}
\end{comment}


\subsection{Organization} Write the organization of the document here. For example: Section~\ref{sec:introduction} gives the overview of the project, Section~\ref{sec:projectmanagement} describes how the project and the resources are managed. .........Finally, the conclusion and the pointers to the future work are outlined in Section~~\ref{sec:cfw}.

\clearpage