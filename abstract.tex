\begin{abstract}
The University of Chittagong’s online payment and attendance system is the objective of this project report. A large number of students at the university pay all university fees using bank drafts to the institution’s accounts at a particular bank branch that does not allow for the use of internet payment methods. Moreover, a manual attendance system is still being practised here. Both of these analogue systems are inefficient. Particularly during examination seasons, when the majority of students are required to pay examination fees. It is marked by lengthy lines, excessive waiting on the part of students, and congestion at the banks where payments are made throughout this procedure. On the other hand, the current manual attendance system consumes a significant amount of time every day. Against this backdrop, we began work on a project to create an alternative payment and attendance system that would allow students to pay and show up for the class online. This method ensures that all students are acquainted with the online payment processes. Additionally, taking attendance online will save time, and classes will be more effective. 


To maintain the system development process, we used the Software Development Life Cycle (SDLC) and the Scrum method to help team members work together. The system was developed using a JavaScript-based framework called ”React”, which includes Cascading Style Sheets(CSS) for the front-end and ”ExpressJs” for the back-end, as well as an Apache web server and a MySQL database server. Testing and validation of the system were also carried out by enabling users to engage with it while interacting with test data. For the time being, the system is solely capable of handling the payment and attendance systems. Our system can develop this system in the future to include numerous online systems such as the No Objection Certificate (NOC), the Student Management System, the Employee Management System, and so on. The project’s outcome is an online payment and attendance system for the University of Chittagong, which alleviates the long-standing challenges associated with the university’s present ways of payment and the time-consuming manual attendance method.

%Explain the following points: Why are you doing this database project? What is the problem you choose? Why does it motivate you? What are current problems faced by the stack-holders? What solution will your system provide? What are the process you will use to develop your solution? The significance of your project, limitation and future work in short%


\end{abstract}
%\clearpage